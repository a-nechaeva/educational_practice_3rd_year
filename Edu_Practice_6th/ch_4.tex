\chapter{Построение сети из двух связанных нейронов}
\label{ch:chap7}

Будем рассматривать гетерогенную сеть из двух нейронов: одного регулярно-спайкового(RS) и одного быстро-спайкового(FS). Введем переменную $\sigma$, характеризующую силу связи между нейронами сети. Запишем дифференциальные уравнения для моделирования динамики связанных нейронов

\begin{equation}
	\begin{cases}
			\frac{dv_1}{dt} = 0.04v_1^2+5v_1+140-u_1+I_1 + \sigma(v_2 - v_1),\\
			\frac{du_1}{dt} = a(bv_1-u_1),\\
			\frac{dv_2}{dt} = 0.04v_2^2+5v_2+140-u_2+I_2 + \sigma(v_1 - v_2),\\
			\frac{du_2}{dt} = a(bv_2-u_2),\\
	\end{cases}
\end{equation}

вспомогательный сброс

\begin{equation}
	\text{если } v_i \geq 30 \, \, mV , \text{ то } \begin{cases}
		v_i \leftarrow c_i\\
		u_i \leftarrow u_i+d_i,
	\end{cases}
\end{equation}

Под синхронизацией в данной задачей будем понимать согласованное во времени функционирование двух объектов (нейронов сети)\cite{sem}. Для того, чтобы оценить уровень синхронизации в зависимости от заданной силы связи $\sigma$ введем функцию $s(\sigma)$, значение которой вычисляется по формуле

\begin{equation}
	s(\sigma) = \frac{1}{n} \sum \limits_{t = t_1}^{t_n} (v_1(t) - v_2(t))^2
\end{equation}

\endinput