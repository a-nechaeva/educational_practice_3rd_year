\chapter{Заключение}
\label{ch:chap6}
В ходе выполнения работы была изучена математическая модель нейрона Ижикевича. Рассмотрено влияние параметров модели на характеристики моделируемого нейрона. Выполнено численное моделирование изменения мембранного потенциала нейрона, а также вспомогательной переменной, отвечающей за восстановление мембранного потенциала. Теоретические данные были подтверждены проведенным моделированием, кроме того, было выяснено, что увеличение значения входного тока приводит к росту частоты спаек во всех рассмотренных наборах параметров модели.

 В последних главах работы была построена гетерогенная сеть из двух связанных нейронов. Выполнено моделирование динамики нейронов сети при различных значениях силы связи, совместно с определением уровня синхронизации нейронов сети. Было выяснено, что рост силы связи приводит к увеличению степени синхронизации динамики нейронов в сети.

\endinput