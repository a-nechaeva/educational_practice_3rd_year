\chapter{Математическая модель биологического нейрона}
\label{ch:chap2}

\section{Модель Ижикевича}
Нейронная модель Ижикевича описывает изменение электрического потенциала в мембране нейрона в зависимости от тока, протекающего через ионные каналы мембраны. Изменения электрического потенциала представлены следующими дифференциальными уравнениями

\begin{equation}
	\begin{cases}
		 \frac{dv}{dt} = 0.04v^2+5v+140-u+I,\\
		\frac{du}{dt} = a(bv-u),
	\end{cases}
\end{equation}
вспомогательный сброс

\begin{equation}
    \text{если } v \geq 30 \, \, mV , \text{ то } \begin{cases}
        v \leftarrow c\\
        u \leftarrow u+d,
    \end{cases}
\end{equation}

где $v$ -- мембранный потенциал нейрона, $u$ -- вспомогательная переменная, которая восстанавливает мембранный потенциал: отвечает за активацию ионных токов $K^+$ и инактивацию ионных токов $Na^+$ и обеспечивает отрицательную обратную связь с $v$ \cite{dhamo2021efficient}.

Спайк (или потенциал действия) -- это кратковременное (~1-2 мс) резкое увеличение мембранного потенциала нейрона $(v)$, которое возникает при достижении порогового значения и служит для передачи информации между нейронами. В модели Ижикевича спайки формализованы через пороговый сброс переменных $v$ и $u$. 


\section{Параметры модели}


Параметр $a$ отвечает за временнной масштаб переменной восстановления $u$. Чем меньше значение $a$, тем медленнее восстановление. Значение по умолчанию  $a=0.02$ \cite{dhamo2021efficient}.

Параметр $b$ характеризует чувствительность переменной восстановления $u$ к подпороговым колебаниям мембранного потенциала $v$. При увеличении значения параметра $b$ возрастает и сила связи $v$ и $u$, что приводит к возможным подпороговым колебаниям и низкопороговой динамике спайков. Значение по умолчанию $b = 0.2$ \cite{dhamo2021efficient}.

Параметр $c$ соответствует значению сброса мембранного потенциала $v$ после скачка, вызванного быстрыми высокопороговыми проводимостями $K^+$. Значение по умолчанию $c=-65$ мВ \cite{dhamo2021efficient}.

Параметр $d$ характеризует сброс переменной восстановления $u$ после скачка, вызванный медленными высокопороговыми проводимостями $Na^+$ и $K^+$. Значение по умолчанию $d=2$ \cite{dhamo2021efficient}. 

С помощью вариации параметров $a$, $b$, $c$, $d$ можно добиться моделирования различных типов нейронов.



\section{Виды нейронов}

Рассмотрим модели нейронов, которые можно получить при определенных значениях параметров.

\subsection{Регулярно-спайковые (Regular spiking)}

Воссоздают динамику стандартных возбуждающихся нейронов, например, пирамидных нейронов коры. Параметры модели Ижикевича в данном случае: $a=0.02$, $b=0.2$, $c = -65$, $d = 8$. Возможные начальные условия: $v_0 = -65$, $u_0 = b \cdot v_0 = -13$, значение тока $I \in [5; 15]$ нА. При постоянном стимуле генерируют регулярные спайки с постепенным увеличением интервала между спайками.

\subsection{Быстро-спайковые (Fast spiking)}

Моделируют быстрые ГАМК-ергические интернейроны, например, корзинчатые клетки. Параметры модели Ижикевича: $a=0.1$, $b=0.2$, $c = -65$, $d = 2$. Начальные условия: $v_0 = -70$, $u_0 = b \cdot v_0 = -14$, ток $I \in [10; 20]$ нА. В данном случае формируются высокочастотные спайки, причем увеличения интервала между спайками на происходит.


\subsection{Низкопороговые спайковые (Low-threshold spiking)}

Моделируют низкопороговые интернейроны, например, соматостатин-положительные. Параметры модели Ижикевича: $a=0.02$, $b=0.25$, $c = -65$, $d = 2$. Начальные условия: $v_0 = -70$, $u_0 = b \cdot v_0 = -17.5$, ток $I \in [5; 10]$ нА. С точки зрения динамики характерна высокая частота спайков в начале и быстрое увеличение интервала между спайками с ростом времени.


\subsection{Резонансные (Resonator)}

Симулируют нейроны с резонансными свойствами, например, таламические релейные нейроны. Параметры модели Ижикевича: $a=0.1$, $b=0.26$, $c = -65$, $d = 2$. Начальные условия: $v_0 = -65$, $u_0 = b \cdot v_0 = -16.9$, ток $I \in [5; 15]$ нА. Генерация спайков только на определенных частотах стимула.


\subsection{Внутренне разрывные (Intrinsically bursting)}
Нейроны с аутохтонными всплесками, например, в таламусе. Параметры модели Ижикевича: $a=0.02$, $b=0.2$, $c = -55$, $d = 4$. Начальные условия: $v_0 = -60$, $u_0 = b \cdot v_0 = -12$, ток $I \in [5; 15]$. Вначале высокочастотный участок спайков, затем формируются автоматические пачки спайков даже без стимула.


\subsection{Чаттерные (Chattering)}
Описываемая динамика близка быстроразряжающимся кортикальным нейронам, находящимся в зрительной коре и других областях неокортекса. Параметры модели Ижикевича: $a=0.02$, $b=0.2$, $c = -50$, $d = 2$. Начальные условия: $v_0 = -65$, $u_0 = b \cdot v_0 = -13$, ток $I \in [5; 10]$. Характерно формирование высокочастотных пачек спаек.





\endinput